\documentclass[conference]{IEEEtran}
\usepackage[utf8]{inputenc}
\usepackage[spanish]{babel}
\usepackage{hyperref}

\title{TuTurismo Neiva: Fomentando la Cultura y el Turismo a Través de la Innovación Tecnológica}

\author{
    \IEEEauthorblockN{Valentina Silva Garrido,\\ Mariana González Calderón, \\Maydy Viviana Conde Ladino,\\
    Dylan Santiago Narváez Pinto,\\ Isabela Gutiérrez Córdoba, \\Iván Andrés Murcia Epia}
    \IEEEauthorblockA{
        Neiva, Huila}
}

\begin{document}

\maketitle

\section{Resumen}
Este artículo tiene como objetivo presentar la aplicación TuTurismo Neiva, diseñada para promover el turismo en la ciudad de Neiva a través de una plataforma digital interactiva. La app permite a los usuarios explorar, descubrir y valorar los sitios turísticos y monumentos de la ciudad, integrando la historia local con herramientas modernas de evaluación y recomendación. La metodología empleada incluye enfoques ágiles como Scrum y XP, junto con herramientas gráficas como Canva para el diseño de la interfaz. Los resultados muestran una mayor visibilidad de sitios menos conocidos y una interacción mejorada con el patrimonio cultural. Además, el sistema PQRSFD permite a los usuarios expresar opiniones y sugerencias, fomentando un proceso de mejora continua. En conclusión, TuTurismo Neiva promueve tanto el conocimiento cultural como el desarrollo económico y social de la ciudad.


\section{Palabras claves}
Aplicación móvil, turismo, cultura, Neiva, innovación, historia, PQRSFD.

\section{Introducción}
El turismo digital se ha convertido en una herramienta clave para conectar a turistas y comunidades locales, enriqueciendo las experiencias culturales. En este contexto, la aplicación *TuTurismo Neiva* busca aprovechar las tecnologías modernas para destacar los atractivos turísticos de Neiva, mejorando su visibilidad y fomentando la interacción con su patrimonio cultural.

\section{Metodología}
El desarrollo de la aplicación siguió un enfoque ágil utilizando las metodologías Scrum y XP, lo que permitió un proceso iterativo enfocado en funcionalidades específicas como la geolocalización y la integración del sistema PQRSFD. Herramientas como Canva facilitaron el diseño de interfaces intuitivas y atractivas, mientras que pruebas de usabilidad con usuarios locales y turistas ayudaron a refinar la experiencia.

\section{Resultados}
La aplicación ha registrado más de 50 sitios turísticos, valorados por los usuarios a través de un sistema de calificación que resalta tanto lugares conocidos como menos visitados. El sistema PQRSFD ha recopilado información clave para la mejora continua de los servicios ofrecidos.

\section{Conclusiones}
*TuTurismo Neiva* representa un paso significativo en la promoción del turismo local a través de la tecnología, combinando metodologías ágiles y herramientas innovadoras. Se recomienda expandir la aplicación a nivel nacional y añadir funcionalidades como itinerarios personalizados para maximizar su impacto.

\section*{Agradecimientos}
Agradecemos a todos los colaboradores que participaron en el desarrollo y prueba de la aplicación, así como a las comunidades locales que han contribuido con sus valiosos comentarios.

\end{document}
